\section{DeepInf: Social influence prediction}
\subsection{Introduction}
\paragraph{} Deepinf focus on the prediction of user-level social influence. It aims to \textbf{predict the action status of a user given the action}.
It is a deep learning based framework to represent both influence dynamics and network structures into a latent space and 
tries to minimize the negative likelihood that was defined in the section 1.
\subsection{Model framework}
\subsubsection{Sampling Near Neighbour}
\paragraph{} Give a user $v$, a r-ego network $\mathcal{G}_v^r$ is extracted using breadth-first search (BFS) starting from 
user $v$. However, $\mathcal{G}_v^r$ may have different size due to the small world property in the social network. Since most
deep learning models expects fixed size data, the graph $\mathcal{G}_v^r$ can be sampled to fixed size.
\paragraph{} For sampling a fixed size graph \textbf{random walk the restart} (RWR) was used. RWR algorithm is defined as
following steps.
\begin{itemize}
    \item Start random walks from either the ego user $v$ or one of her active neighbors randomly.
    \item The random walk iteratively travels to its neighborhood with the probability that is 
    proportional to the weight of each edge.
    \item At each step, the walk is assigned a probability to return to the starting node, that is, 
    either the ego user $v$ or one of $v$’s active neighbors.
    \item Run the algorithm until a fixed number of vertices denoted by $\hat{\Gamma_v^r}$ with $\hat{|\Gamma_v^r|}$=n. 
\end{itemize}
\paragraph{} After running this algorithm, a sub-graph $\hat{\mathcal{G}_v^r}$ and denote $\hat{S}_v^t$=
\{ $s_u^t :u \in \hat{\Gamma_v^r}$ \} be the action statuses of $v$'s sampled neighbours. Therefore we re-define the optimization
objective in section 1 as:
\begin{equation}
    \mathcal{L}(\theta) = -\sum_{i=1}^N \log(P_{\theta}(s_{v_i}^{t_i+\Delta t}|\hat{\mathcal{G}_{v_i}^{t}},s_{v_i}^{t_i}))
\end{equation}
\subsubsection{Embedded layer}
\subsubsection{Instane Normalization}
\subsubsection{Input Layer}
\subsubsection{GCN}
\subsection{Evaluation Metrics}